\documentclass[11pt, a4paper]{article}


\usepackage{wrapfig}
\usepackage{graphicx}
\usepackage[T1]{fontenc}
\usepackage[polish]{babel}
\usepackage[utf8]{inputenc}
\usepackage[font=footnotesize, labelfont=bf]{caption}
\usepackage{csquotes}
\usepackage{placeins}
\usepackage{booktabs}
\usepackage{siunitx}
\usepackage{amsmath}
\usepackage[margin=1in]{geometry}
\usepackage{hyperref}
\hypersetup{
  colorlinks,
  citecolor=black,
  filecolor=black,
  linkcolor=black,
  urlcolor=black
}
\usepackage[backend=biber, sorting=ynt]{biblatex}
\addbibresource{draft.bib}

\newcommand{\code}[1]{\texttt{#1}}
\linespread{1.3}

\tolerance=1
\emergencystretch=\maxdimen
\hyphenpenalty=10000
\hbadness=10000

\title{Bibliography
management:
\texttt{biblatex}
package}


\author{Krzytsztof
Wiśniewski}
\date{
}


\begin{document}
  \begin{titlepage}
    \centering


    \Large \textbf{UNIWERSYTET GDAŃSKI}\\ \textbf{WYDZIAŁ MATEMATYKI, FIZYKI I
    INFORMATYKI}

    \vspace{2.5cm}


    \large \textbf{Krzysztof Wiśniewski}\\ \textbf{numer albumu: 274276}

    \vspace{1.5cm}
    \raggedright \small Kierunek studiów: Bioinformatyka\\ Specjalność: Ogólna

    \vspace{1.5cm}


    \centering
    \Large \textbf{Optymalizacja oprogramowania do detekcji splątania kwantowego}

    \vfill


    \raggedleft \normalsize Praca licencjacka\\ wykonana\\ pod kierunkiem\\ dr hab. Marcin
    Wieśniak, prof. UG\\

    \vfill


    \centering
    \large Gdańsk 2023
  \end{titlepage}
  \newpage


  \tableofcontents
  \newpage


  \begin{sloppypar}
    \begin{abstract}
      W tej pracy przeprowadzam analizę efektywności metod optymalizacji czasu pracy programu
      CSSFinder służącego do detekcji splątania kwantowego. Podejmowane przeze mnie
      wysiłki skupiają się w głównej mierze na doborze lepszych narzędzi pozwalających
      na wydajniejsze prowadzenie dużych ilości obliczeń macierzowych. Rozważane będą skutki
      re-implementacji w języku Python z wykorzystaniem bibliotek NumPy\cite{NumPy_Article}\cite{NumPy_Doc},
      Numba\cite{Numba_Article}\cite{Numba_Doc}, Cython\cite{Cython_The_Best_Of_Both}\cite{Cython_Org}
      oraz re-implementacja w języku Rust\cite{Rust_Programming_Language} z
      wykorzystaniem skrzyni\footnote{ang. crate, określenie na bibliotekę-pakiet w ekosystemie
      języka Rust.}. Każda z implementacji została przetestowana na specjalnie dobranym
      zestawie danych, a wyniki są na bieżąco omawiane. Pod koniec podsumowuję wady i
      zalety poszczególnych rozwiązań i rozważam w jakich scenariuszach najlepiej się
      one sprawują.
    \end{abstract}

    \section{Wstęp}


    \subsection{Dlaczego Python?}


    Na przestrzeni ostatnich 20 lat język Python, stworzony przez Guido van Rossum,
    zanotował intensywny wzrost popularności. Pokazują to liczne zestawienia, w tym
    zestawienie najczęściej wykorzystywanych języków programowanie na GitHub'ie\cite{GitHub_Top_languages},
    w którym Python w roku 2022 zajął 2 miejsce, czy też zestawienie TIOBE Index\cite{TIOBE_Software_Index},
    uznające ten język za obecnie najbardziej rozpowszechniony pośród doświadczonych programistów
    (Maj 2023).

    \FloatBarrier
    \begin{figure}[ht]
      \centering
      \includegraphics
      [width=0.75\textwidth]{"resources/images/python_language_results.png"}
      \caption{Ilość wyników zwróconych przez wyszukiwarkę Google Scholar dla zapytania 'python language' z podziałem na rok opublikowania zasobu w przestrzeni publicznej.}
    \end{figure}
    \FloatBarrier

    Niestety, interpretowany kod, napisany w Pythonie, pomimo licznych zalet, posiada
    również dotkliwą wadę - pod względem wydajności znacząco odstaje od kompilowanych
    języków programowania (C\cite{C_vs_Python}, C++\cite{Cpp_vs_Python}, Rust\cite{Rust_vs_Python}).
    Jednak, podejmując odpowiednie wysiłki, możliwe jest aby programy, których kluczowa logika
    została napisana w języku Python, zbliżały się wydajnością do odpowiedników
    przetłumaczonych na kod maszynowy. Taki stan rzeczy czyni z języka Python bardzo wygodny
    język do prototypownia w procesie wytwarzania nowych rozwiązań programistycznych.

    \subsection{Cel pracy}


    Praca ta ma na celu eksplorację wybranych metod optymalizacji czasu wykonania oprogramowania
    CSSFinder oraz weryfikację uzyskanych zmian wydajności programu. W dalszej jej
    części zaprezentuję wyniki testów wydajności i przeanalizuję specyfikę
    poszczególnych metod maksymalizacji wydajności. Podejście do redukcji czasu wykonywania
    programu zaprezentowane w tej pracy można zastosować dla większości oprogramowania,
    napisanego w języku Python, które koncentruje się na wykonywaniu dużych ilości obliczeń
    macierzowych.

    Do przeprowadzenia takich analiz konieczne było wielokrotne re-implementowanie części
    obliczeniowej programu. Funkcjonalny kod opisywany w tej pracy dostępny jest w
    repozytoriach Gita\cite{Git_Com} w serwisie GitHub \cite{CSSFinder_New}\cite{CSSFinder_New_Numpy}\cite{CSSFinder_New_Rust}.
    W skutek prac projektowych utworzyłem również grupę publicznych pakietów, które można
    pobrać z serwisu PyPI:

    \begin{itemize}
      \item cssfinder\cite{CSSFinder_New_PyPI}

      \item cssfinder\_backend\_numpy\cite{CSSFinder_New_Numpy_PyPI}

      \item cssfinder\_backend\_rust\cite{CSSFinder_New_Rust_PyPI}
    \end{itemize}

    Zainstalowanie ich jest możliwe przy pomocy menadżera pakietów języka Python, np.
    pip\cite{PIP}. Pakiety są kompatybilne z implementacją CPython w wersjach 3.8 - 3.10
    i były testowane na systemach Windows (10), Linux (Ubuntu 22.04) oraz macOS (12).

    \subsection{Program CSSFinder}


    Program CSSFinder, którego autorem jest dr hab. Marcin Wieśniak, prof. UG, z
    wydziału Matematyki, Fizyki i Informatyki Uniwersytetu Gdańskiego. Kod implementuje
    wyspecjalizowany wariant algorytmu zaproponowanego przez E. Gilberta\cite{Lindemann_Gilbert}
    który służy do znajdowania odległości pomiędzy punktem, a zbiorem wypukłym.
    Oprogramowanie jest przeznaczone do detekcji splątania kwantowego\cite{MW_Hilbert_Schmidt_distance}\cite{MW_Variational_approach}\cite{MW_Gilbert_Quantum_Entanglement}
    poprzez analizę macierzy gęstości opisujących układy kubitów\footnote{Potencjalnie
    również ku$d$itów, natomiast tego typu dane nie będą analizowane w tej pracy.}.
    Algorytm ten wielokrotnie, z sukcesem, był wykorzystany do analizy problemów z
    dziedziny fizyki kwantowej\cite{MW_Hilbert_Schmidt_distance} przy okazji również pokonując
    rozwiązania bazujące na uczeniu maszynowym\cite{MW_56_Year_Algorithm}.

    Oryginalna implementacja wykorzystuje język Python oraz bibliotekę NumPy. Posiada ona
    4 różne tryby pracy, dedykowane do rozwiązywania różnych problemów, oznaczane kolejno
    cyframi:

    \begin{enumerate}
      \item pełna separowalność stanu $n$ ku$d$itów\footnote{ang. full separability of an
        $n$ qu$d$it state},

      \item separowalność stanu dwudzielnego\footnote{ang. separability of a bipartite state}

      \item rzeczywiste 3-częściowe uwikłanie stanu trzech ku$d$itów \footnote{ang.
        genuine 3-partite entablement of a 3-quDit state}

      \item rzeczywiste 4-częściowe uwikłanie stanu trzech ku$d$itów \footnote{ang.
        genuine 4-partite entablement of a 3-quDit state}
    \end{enumerate}

    Dodatkowo pozwala na podanie macierzy symetrii oraz macierzy projekcji układu.

    \subsection{Modularyzacja}


    Podczas procesu optymalizacji planowałem wypróbować liczne rozwiązania, które wymagały
    zasadniczych zmian w algorytmie. Jednocześnie część programu odpowiadająca za
    interakcję z użytkownikiem i ładowanie zasobów miała pozostawać taka sama. Zdecydowałem
    więc że tworzony przeze mnie kod musi być modularny, aby uniknąć duplikacji
    wspólnych elementów. Tak też program został podzielony na dwie części: główną (core),
    z interfejsem użytkowników i narzędziami pomocniczymi oraz część implementującą algorytm
    (backend). Korpus jest w całości napisany w języku Python i wykorzystuje wbudowany w
    ten język mechanizm importowania bibliotek w celu wykrywania i ładowania
    implementacji algorytmu. Dane macierzowe w obrębie korpusu przechowywane są jako obiekty
    ndarray z biblioteki NumPy, ze względu na uniwersalność w świecie bibliotek do
    obliczeń tensorowych. Pozwala to na proste podmiany implementacji o dowolnie różnym pochodzeniu,
    w tym implementacje w językach kompilowanych. Uprościło to znacznie proces weryfikacji
    zmian w zachowaniu programu i przyspieszyło proces tworzenia kolejnych implementacji,
    jako że kod interfejsu programistycznego jest mniej pracochłonny niż kod pozwalający
    na interakcję z użytkownikiem.
    \newpage


    \section{Metody}


    \subsection{Kompilacja AOT}


    Kompilacja AOT (Ahead Of Time) to proces tłumaczenia jednej reprezentacji programu (na
    przykład w języku programowania wysokiego poziomu) na inną (na przykład kod
    maszynowy) przed rozpoczęciem pracy kompilowanego programu.

    Obecnie najpowszechniej używana implementacja języka Python, CPython, posiada możliwość
    korzystania z bibliotek współdzielonych (.so - Linux, .dll/.pyd - Windows) które powstały
    w skutek kompilacji kodu wysokiego poziomu. Dostęp do funkcji zawartych w takich
    bibliotekach można uzyskać na kilka sposobów:

    \begin{enumerate}
      \item Przy pomocy API modułu ctypes\cite{Python_ctypes}. Pozwala ono opisać interfejs
        funkcji obcej (tj. takiej która została napisana w języku niższego poziomu i skompilowana
        do kodu maszynowego) i wywołać tak opisaną funkcję.

      \item Poprzez zawarcie w bibliotece odpowiednio nazwanych symboli, automatycznie
        rozpoznawanych przez interpreter języka Python. Takie biblioteki określa się mianem
        modułów rozszerzeń \cite{Extending_Python_With_C_Cpp}. W tym przypadku warto
        dodać, że pomimo, że oficjalna dokumentacja wspomina tylko o językach C i C++, natomiast
        powstały biblioteki które pozwalają wykorzystać w łatwy sposób wiele innych języków
        programowania, takich jak Rust przy pomocy Py03\cite{PyO3} lub GO z użyciem biblioteki
        gopy\cite{gopy}.

      \item Wykorzystując bibliotekę Cython\cite{Cython_Org}\cite{Cython_The_Best_Of_Both}.
        Oferuje ona dedykowany język, o tej samej nazwie, który jest nadzbiorem języka
        Python, który rozszerza jego składnię o możliwość statycznego typowania.
        Biblioteka zawiera transpilator, zdolny przetłumaczyć dedykowany język na C/C++,
        a następnie, wykorzystując osobno zainstalowany kompilator, skompilować do kodu
        maszynowego.

      \item Kompilując kod pythona z użyciem biblioteki mypyc\cite{mypyc}. Ta, podobnie
        do biblioteki Cython, również zawiera transpilator, natomiast zamiast korzystać
        z dedykowanego języka, opiera się on na dodanych w Pythonie 3.5\cite{Python_3_5}
        (PEP 484\cite{PEP_484} i PEP 483\cite{PEP_483}), adnotacjach typów. Jest on
        rozwijany obok projektu mypy - pakietu do statycznej analizy typów dla języka
        Python, również opartej na adnotacjach typów\cite{mypy}.
    \end{enumerate}

    Ponieważ w każdym z wymienionych przypadków, kod niższego poziomu jest kompilowany
    przed dostarczeniem do użytkownika, pozwala to na wykorzystanie zaawansowanych możliwości
    automatycznej optymalizacji dostarczanych przez współczesne kompilatory, na przykład
    LLVM, które jest sercem implementacji clang (język C++) oraz rustc (język Rust).
    Wiele bibliotek korzysta z mieszanek wymienionych powyżej metod, w tym cieszące się dużą
    popularnością NumPy, CuPy, Tensorflow, czy PyTorch. Dwie ostatnie biblioteki koncentrują
    się w głównej mierze na uczeniu maszynowym i głównie pod tym kontem są optymalizowane.
    Ich interfejsy są bardzo zbliżone do NumPy i CuPy, ale brakuje w nich niektórych narzędzi,
    które nie znajdują zastosowania w dziedzinie sztucznej inteligencji. W dalszej części
    pracy intensywnie wykorzystywana będzie biblioteka NumPy. Niestety, ze względu na ograniczenia
    czasowe oraz wstępne przewidywania dotyczące wydajności\footnote{CuPy jest
    odpowiednikiem NumPy który wykorzystuje do obliczeń GPU. Z tego względu radzi sobie wyśmienicie
    z operacjami na dużych macierzach, natomiast najprawdopodobniej macierze tutaj
    rozważane są zbyt małe aby uzyskać wzrost wydajności\cite{CPU_VS_GPU}. Jednocześnie
    pomimo podobieństwa do NumPy, biblioteka ta różni się i posiada problematyczne zależności
    (CUDA) co czyni adaptację kodu czasochłonną.} biblioteka CuPy nie wzięta pod uwagę.

    Ponadto w zestawieniu pojawi się implementacja napisana w języku Rust, wykonująca
    operacje macierzowe w oparciu o bibliotekę Ndarray\cite{Ndarray}. Komunikacja
    pomiędzy interpreterem Pythona, a biblioteką oparta została o rozwiązanie opisane w punkcie
    drugim, dzięki wspomnianej tam bibliotece PyO3\cite{PyO3}. Jest to wymiernie
    reprezentatywny przykład tego jaką wydajność można uzyskać tworząc kod w typowym
    niskopoziomowym języku programowania, posiadającym relatywnie niskopoziomową kontrolę
    nad pamięcią. Dodatkowo w zestawieniach pojawi się wariant tej implementacji który
    dodatkowo będzie wykorzystywał bibliotekę OpenBLAS\cite{OpenBLAS}\cite{Ndarray} do
    operacji mnożenia macierzowego.

    \subsection{Kompilacja JIT}


    Kompilacja JIT to proces tłumaczenia jednej reprezentacji programu (na przykład w języku
    programowania wysokiego poziomu) na inną (na przykład kod maszynowy) po rozpoczęciu
    pracy programu. Zazwyczaj wymaga to aby program rozpoczynał pracę w trybie interpretowanym,
    a następne kompilował sam siebie i przechodził w tryb wykonywania skompilowanego
    kodu.

    W momencie pisania tej pracy istnieją dwa szeroko dostępne i aktywnie utrzymywane
    narzędzia oferujące kompilację JIT dla języka Python.

    Pierwszym z nich jest pełna alternatywna implementacja języka Python - PyPy\cite{PyPy_Home_Page}.
    Wykonywana przez nią kompilacja JIT działa on na podobnej zasadzie do uprzednio
    wymienionych - śledzi cały kod który wykonuje i automatycznie decyduje które
    fragmenty skompilować do kodu maszynowego\cite{PyPy_JIT}. Niestety, posiada ona zasadniczą
    wadę - jej interfejs binarny\footnote{ang. ABI - Application Binary Interface} oraz
    programistyczny\footnote{ang. API - Application Programming Interface.} różni się od
    CPythona, a większość pakietów które normalnie wykorzystują moduły rozszerzeń nie oferuje
    pre-kompilowanych pakietów dla PyPy. Powoduje to że instalacje takich pakietów są
    bardzo czasochłonne i obecności kompilatora na urządzeniu docelowym. Dodatkowo, pre-kompilowany
    kod nie czerpie żadnych korzyści z kompilatora JIT zawartego w PyPy. Problemy te powodują,
    że PyPy nadaje się głównie do wykonywania aplikacji napisanych w czystym języku
    Python.

    Drugim narzędziem jest biblioteka Numba\cite{Numba_Article}\cite{Numba_Doc}. Ona, w
    przeciwieństwie do PyPy, wymaga aby fragmenty kodu, które mają być skompilowane, miały
    postać funkcji oznaczonych dedykowanymi dekoratorami\footnote{Obecnie dostępny jest
    też dekorator pozwalający na kompilację klas, niestety jest on niestabilny i nie radzi
    sobie w wielu sytuacjach.}. Została ona również zaprojektowana aby dobrze współgrać
    z biblioteką NumPy. Jej zastosowanie z założenia ma generować wzrost wydajności nawet
    w sytuacjach gdy kod programu bardzo mocno eksploatuje możliwości biblioteki NumPy.

    Z uprzednio wymienionych względów dotyczących preferowanych zastosowań powyższych rozwiązań
    w dalszej części będę próbował wykorzystać bibliotekę Numba, natomiast pominę
    możliwość skorzystania z PyPy.

    \newpage


    \subsection{Dane testowe}


    Podczas pomiarów konsekwentnie wykorzystywałem ten sam zestaw macierzy gęstości, aby
    móc wygodnie porównywać wyniki wydajności poszczególnych implementacji. W dalszej
    części pracy będę wielokrotnie odnosił się do tych macierzy posługując się symbolem
    $\rho$ z liczbą w indeksie dolnym. Liczba ta będzie wskazywać na konkretną z wymienionych
    poniżej macierzy.

    \FloatBarrier
    \begin{figure}[ht]
      \centering
      \setcounter{MaxMatrixCols}{33}
      \[
        \rho_{0}= \left[
        \begin{smallmatrix}
          0.25  & 0 & 0.25  & 0 & 0 & 0 & 0 & 0 & 0 & 0 & 0 & 0 & 0 & 0 & 0 & 0 & 0 & 0 & 0 & 0 & 0 & 0 & 0 & 0 & 0.25  & 0 & 0 & 0 & 0 & 0 & 0 & -0.25 \\
          0     & 0 & 0     & 0 & 0 & 0 & 0 & 0 & 0 & 0 & 0 & 0 & 0 & 0 & 0 & 0 & 0 & 0 & 0 & 0 & 0 & 0 & 0 & 0 & 0     & 0 & 0 & 0 & 0 & 0 & 0 & 0     \\
          0.25  & 0 & 0.25  & 0 & 0 & 0 & 0 & 0 & 0 & 0 & 0 & 0 & 0 & 0 & 0 & 0 & 0 & 0 & 0 & 0 & 0 & 0 & 0 & 0 & 0.25  & 0 & 0 & 0 & 0 & 0 & 0 & -0.25 \\
          0     & 0 & 0     & 0 & 0 & 0 & 0 & 0 & 0 & 0 & 0 & 0 & 0 & 0 & 0 & 0 & 0 & 0 & 0 & 0 & 0 & 0 & 0 & 0 & 0     & 0 & 0 & 0 & 0 & 0 & 0 & 0     \\
          0     & 0 & 0     & 0 & 0 & 0 & 0 & 0 & 0 & 0 & 0 & 0 & 0 & 0 & 0 & 0 & 0 & 0 & 0 & 0 & 0 & 0 & 0 & 0 & 0     & 0 & 0 & 0 & 0 & 0 & 0 & 0     \\
          0     & 0 & 0     & 0 & 0 & 0 & 0 & 0 & 0 & 0 & 0 & 0 & 0 & 0 & 0 & 0 & 0 & 0 & 0 & 0 & 0 & 0 & 0 & 0 & 0     & 0 & 0 & 0 & 0 & 0 & 0 & 0     \\
          0     & 0 & 0     & 0 & 0 & 0 & 0 & 0 & 0 & 0 & 0 & 0 & 0 & 0 & 0 & 0 & 0 & 0 & 0 & 0 & 0 & 0 & 0 & 0 & 0     & 0 & 0 & 0 & 0 & 0 & 0 & 0     \\
          0     & 0 & 0     & 0 & 0 & 0 & 0 & 0 & 0 & 0 & 0 & 0 & 0 & 0 & 0 & 0 & 0 & 0 & 0 & 0 & 0 & 0 & 0 & 0 & 0     & 0 & 0 & 0 & 0 & 0 & 0 & 0     \\
          0     & 0 & 0     & 0 & 0 & 0 & 0 & 0 & 0 & 0 & 0 & 0 & 0 & 0 & 0 & 0 & 0 & 0 & 0 & 0 & 0 & 0 & 0 & 0 & 0     & 0 & 0 & 0 & 0 & 0 & 0 & 0     \\
          0     & 0 & 0     & 0 & 0 & 0 & 0 & 0 & 0 & 0 & 0 & 0 & 0 & 0 & 0 & 0 & 0 & 0 & 0 & 0 & 0 & 0 & 0 & 0 & 0     & 0 & 0 & 0 & 0 & 0 & 0 & 0     \\
          0     & 0 & 0     & 0 & 0 & 0 & 0 & 0 & 0 & 0 & 0 & 0 & 0 & 0 & 0 & 0 & 0 & 0 & 0 & 0 & 0 & 0 & 0 & 0 & 0     & 0 & 0 & 0 & 0 & 0 & 0 & 0     \\
          0     & 0 & 0     & 0 & 0 & 0 & 0 & 0 & 0 & 0 & 0 & 0 & 0 & 0 & 0 & 0 & 0 & 0 & 0 & 0 & 0 & 0 & 0 & 0 & 0     & 0 & 0 & 0 & 0 & 0 & 0 & 0     \\
          0     & 0 & 0     & 0 & 0 & 0 & 0 & 0 & 0 & 0 & 0 & 0 & 0 & 0 & 0 & 0 & 0 & 0 & 0 & 0 & 0 & 0 & 0 & 0 & 0     & 0 & 0 & 0 & 0 & 0 & 0 & 0     \\
          0     & 0 & 0     & 0 & 0 & 0 & 0 & 0 & 0 & 0 & 0 & 0 & 0 & 0 & 0 & 0 & 0 & 0 & 0 & 0 & 0 & 0 & 0 & 0 & 0     & 0 & 0 & 0 & 0 & 0 & 0 & 0     \\
          0     & 0 & 0     & 0 & 0 & 0 & 0 & 0 & 0 & 0 & 0 & 0 & 0 & 0 & 0 & 0 & 0 & 0 & 0 & 0 & 0 & 0 & 0 & 0 & 0     & 0 & 0 & 0 & 0 & 0 & 0 & 0     \\
          0     & 0 & 0     & 0 & 0 & 0 & 0 & 0 & 0 & 0 & 0 & 0 & 0 & 0 & 0 & 0 & 0 & 0 & 0 & 0 & 0 & 0 & 0 & 0 & 0     & 0 & 0 & 0 & 0 & 0 & 0 & 0     \\
          0     & 0 & 0     & 0 & 0 & 0 & 0 & 0 & 0 & 0 & 0 & 0 & 0 & 0 & 0 & 0 & 0 & 0 & 0 & 0 & 0 & 0 & 0 & 0 & 0     & 0 & 0 & 0 & 0 & 0 & 0 & 0     \\
          0     & 0 & 0     & 0 & 0 & 0 & 0 & 0 & 0 & 0 & 0 & 0 & 0 & 0 & 0 & 0 & 0 & 0 & 0 & 0 & 0 & 0 & 0 & 0 & 0     & 0 & 0 & 0 & 0 & 0 & 0 & 0     \\
          0     & 0 & 0     & 0 & 0 & 0 & 0 & 0 & 0 & 0 & 0 & 0 & 0 & 0 & 0 & 0 & 0 & 0 & 0 & 0 & 0 & 0 & 0 & 0 & 0     & 0 & 0 & 0 & 0 & 0 & 0 & 0     \\
          0     & 0 & 0     & 0 & 0 & 0 & 0 & 0 & 0 & 0 & 0 & 0 & 0 & 0 & 0 & 0 & 0 & 0 & 0 & 0 & 0 & 0 & 0 & 0 & 0     & 0 & 0 & 0 & 0 & 0 & 0 & 0     \\
          0     & 0 & 0     & 0 & 0 & 0 & 0 & 0 & 0 & 0 & 0 & 0 & 0 & 0 & 0 & 0 & 0 & 0 & 0 & 0 & 0 & 0 & 0 & 0 & 0     & 0 & 0 & 0 & 0 & 0 & 0 & 0     \\
          0     & 0 & 0     & 0 & 0 & 0 & 0 & 0 & 0 & 0 & 0 & 0 & 0 & 0 & 0 & 0 & 0 & 0 & 0 & 0 & 0 & 0 & 0 & 0 & 0     & 0 & 0 & 0 & 0 & 0 & 0 & 0     \\
          0     & 0 & 0     & 0 & 0 & 0 & 0 & 0 & 0 & 0 & 0 & 0 & 0 & 0 & 0 & 0 & 0 & 0 & 0 & 0 & 0 & 0 & 0 & 0 & 0     & 0 & 0 & 0 & 0 & 0 & 0 & 0     \\
          0     & 0 & 0     & 0 & 0 & 0 & 0 & 0 & 0 & 0 & 0 & 0 & 0 & 0 & 0 & 0 & 0 & 0 & 0 & 0 & 0 & 0 & 0 & 0 & 0     & 0 & 0 & 0 & 0 & 0 & 0 & 0     \\
          0.25  & 0 & 0.25  & 0 & 0 & 0 & 0 & 0 & 0 & 0 & 0 & 0 & 0 & 0 & 0 & 0 & 0 & 0 & 0 & 0 & 0 & 0 & 0 & 0 & 0.25  & 0 & 0 & 0 & 0 & 0 & 0 & -0.25 \\
          0     & 0 & 0     & 0 & 0 & 0 & 0 & 0 & 0 & 0 & 0 & 0 & 0 & 0 & 0 & 0 & 0 & 0 & 0 & 0 & 0 & 0 & 0 & 0 & 0     & 0 & 0 & 0 & 0 & 0 & 0 & 0     \\
          0     & 0 & 0     & 0 & 0 & 0 & 0 & 0 & 0 & 0 & 0 & 0 & 0 & 0 & 0 & 0 & 0 & 0 & 0 & 0 & 0 & 0 & 0 & 0 & 0     & 0 & 0 & 0 & 0 & 0 & 0 & 0     \\
          0     & 0 & 0     & 0 & 0 & 0 & 0 & 0 & 0 & 0 & 0 & 0 & 0 & 0 & 0 & 0 & 0 & 0 & 0 & 0 & 0 & 0 & 0 & 0 & 0     & 0 & 0 & 0 & 0 & 0 & 0 & 0     \\
          0     & 0 & 0     & 0 & 0 & 0 & 0 & 0 & 0 & 0 & 0 & 0 & 0 & 0 & 0 & 0 & 0 & 0 & 0 & 0 & 0 & 0 & 0 & 0 & 0     & 0 & 0 & 0 & 0 & 0 & 0 & 0     \\
          0     & 0 & 0     & 0 & 0 & 0 & 0 & 0 & 0 & 0 & 0 & 0 & 0 & 0 & 0 & 0 & 0 & 0 & 0 & 0 & 0 & 0 & 0 & 0 & 0     & 0 & 0 & 0 & 0 & 0 & 0 & 0     \\
          0     & 0 & 0     & 0 & 0 & 0 & 0 & 0 & 0 & 0 & 0 & 0 & 0 & 0 & 0 & 0 & 0 & 0 & 0 & 0 & 0 & 0 & 0 & 0 & 0     & 0 & 0 & 0 & 0 & 0 & 0 & 0     \\
          -0.25 & 0 & -0.25 & 0 & 0 & 0 & 0 & 0 & 0 & 0 & 0 & 0 & 0 & 0 & 0 & 0 & 0 & 0 & 0 & 0 & 0 & 0 & 0 & 0 & -0.25 & 0 & 0 & 0 & 0 & 0 & 0 & 0.25  \\
        \end{smallmatrix}
        \right]
      \]
      \caption{Macierz $\rho_{0}$.}
      \label{rho-0}
    \end{figure}

    \FloatBarrier

    Pierwsza wymieniana macierz opisuje układ 5 kubitów i posiada wymiary $32\times32$.
    Pomimo że nie zawiera ona wartości, podczas analizy zawsze będzie reprezentowana przez
    macierze zawierające liczby zespolone, ponieważ szczególnie kosztowne obliczeniowo
    części algorytmu wymagają aby części urojone były obecne, co znaczy że usuwanie ich w
    wybranych miejscach nie niesie wymiernych zysków wydajnościowych.

    Następnie w zbiorze macierzy wykorzystywanych jako dane wejściowe znajduje się pięć macierzy
    reprezentujących układy od 2 do 6 kubitów, które przyjmują rozmiary od $4\times 4$
    do $64\times64$. Są one wypełnione zerami poza pierwszym i ostatnim elementem w pierwszej
    i ostatniej kolumnie - te przyjmują wartość $0.5$.

    \FloatBarrier
    \begin{figure}[ht]
      \centering
      \setcounter{MaxMatrixCols}{33}
      \[
        \rho_{n}=
        \begin{bmatrix}
          0.5    & 0      & \hdots & 0      & 0.5    \\
          0      & 0      & \hdots & 0      & 0      \\
          \vdots & \vdots &        & \vdots & \vdots \\
          0      & 0      & \hdots & 0      & 0      \\
          0.5    & 0      & \hdots & 0      & 0.5    \\
        \end{bmatrix}_{(2^{n}\times2^{n})}
      \]
      \caption{Ogólna postać macierzy $\rho_{2}- \rho_{6}$.}
      \label{rho-2-6}
    \end{figure}

    \FloatBarrier

    W tekście macierze te będą oznaczane jako $\rho_{2}$ do $\rho_{6}$, w zależności od
    reprezentowanej liczby kubitów\footnote{Tak więc macierz $\rho_{2}$ ma wymiary $4\times
    4$ i reprezentuje 2 kubity, macierz $\rho_{3}$ ma wymiary $8\times8$ i reprezentuje
    3 kubity, macierz $\rho_{4}$ ma wymiary $16\times16$ i reprezentuje 4 kubity, itd. aż
    do $\rho_{6}$, $64\times64$ .}. Macierze te stanowią wygodny zestaw danych do
    weryfikacji ogólnej charakterystyki zachowania alternatywnych implementacji
    algorytmu, pomimo, że wyniki przy ich pomocy uzyskiwane tak bardzo odbiegają od tych
    uzyskiwanych przy pomocy $\rho_{0}$.

    \newpage


    \subsection{Środowisko testowe}


    Podczas pomiarów wydajności wykorzystywałem każdorazowo to samo środowisko testowe. Do
    chłodzenia CPU wykorzystywane było chłodzenie wodne typu AIO, temperatura w pokoju
    oscylowała w okolicy 25°C, procesor podczas testów wydajności nie doświadczał temperatur
    powyżej 80°C.

    \FloatBarrier
    \begin{table}[ht]
      \centering
      \begin{tabular}{llllll}
    \midrule\midrule
    OS              & Ubuntu 22.04.2 LTS 64-bit         \\
    Kernel          & 5.19.0-42-generic                 \\
    Python          & 3.10.6 64-bit                     \\
    NumPy           & 1.23.5                            \\
    Numba           & 0.56.4                            \\
    Cython          & 3.0.0b1                           \\
    Rust            & 1.68.2 64-bit                     \\\midrule
    CPU             & AMD Ryzen 9 7950X                 \\
    RAM             & 64GB DDR5 5600MHz CL40            \\
    DRIVE           & 512GB SSD GOODRAM CX400 (SATA)    \\\midrule
  \end{tabular}
      \caption{Konfiguracja środowiska testowego.}
      \label{pc-configuration}
    \end{table}
    \FloatBarrier

    \subsection{Profilowanie}


    Podczas prac nad optymalizacją czasu pracy programu kluczowym było stałe zbieranie
    informacji na temat tego które fragmenty kodu pochłaniają najwięcej czasu. Standardowo
    proces zbierania takich danych określa się mianem profilowania i technologie po które
    sięgałem podczas re-implementacji algorytmu posiadają gotowe narzędzia pozwalające na
    skuteczne pozyskiwanie takich danych oraz ich wizualizację.

    Dla kodu w języku Python, implementacja CPython tego języka posiada w bibliotece
    standardowej dwa dedykowane moduły oferujące funkcjonalność profilowania: `profile' i
    `cProfile'. Pierwszy jest zaimplementowany w języku Python, drugi w C. Ponieważ
    drugi z nich posiada mniejszy dodatkowy narzut na procesor, zdecydowałem żeby to na nim
    oprzeć moje analizy. W celu wizualizacji uzyskanych wyników posłużyłem się
    otwartoźródłowym programem snakeviz\cite{Snakeviz_PyPI}.

    Do zbierania informacji na temat charakterystyki pracy kodu napisanego w języku Rust
    wykorzystałem narzędzie perf pochodzące z pakiety linux-tools-5.19.0-42-generic
    pobranego przy pomocy menadżera pakietów apt-get. Do wizualizacji uzyskanych wyników
    wykorzystałem jedno z otwartoźródłowych narzędzi funkcjonujące pod nazwą hotspot\cite{HOTSPOT}.

    \section{Wyniki}


    \subsection{Wstępne profilowanie}


    Prace nad optymalizacją kodu rozpocząłem od wstępnego profilowania pracy programu w
    trybie 1 (ang. full separability of an n-quDit state) przekazując do obliczeń układ 5
    kubitów opisany macierzą $\rho_{0}$ (Patrz rysunek \ref{rho-0}).

    Program wykonywał proces analizy stanu aż do uzyskania 1000 korekcji. Przekazany
    limit liczby iteracji wynosił 2.000.000 i nie został osiągnięty. Podczas pomiarów, program
    wykorzystywał domyślny globalny generator liczb losowych biblioteki NumPy (PCG64\cite{NumpyDefaultGenerator})
    z ziarnem ustawionym na wartość 0.

    \FloatBarrier
    \begin{figure}[ht]
      \centering
      \includegraphics[width=1.0\textwidth]{"resources/profiling_1/graph.png"}
      \caption{Diagram podsumowujący pracę programu wygenerowany przez program snakeviz.}
      \label{pre-prof-perf}
    \end{figure}

    Rysunek \ref{pre-prof-perf} przedstawia diagram, typu Icicle, obrazujący udział
    czasu, pochłoniętego przez wykonywanie poszczególnych funkcji, w całkowitym czasie
    pracy programu. Pierwszy blok od góry to pierwsze wywołanie pochodzące z interpretera.
    Następnie bloki, których opisy zaczynają się od `CSSFinder.py' to wywołania w kodzie
    programu. Najniższe bloki to wywołania do funkcji bibliotek, głównie NumPy. Snakeviz
    automatycznie podejmuje decyzję o nie adnotowaniu bloku gdy opis nie ma szansy
    zmieścić się w obrębie bloku. Aby usunąć z diagramu zbędny szum informacyjny,
    funkcje których wykonywanie zajęło mniej niż 1\% czasu programu były pomijane.

    \FloatBarrier
    \begin{table}[ht]
      \tiny
      \centering
      \begin{tabular}{llllll}
  ncalls  & tottime   & percall   & cumtime   & percall   & filename:lineno(function)         \\
  \midrule\midrule
  1       & 1.431e-05 & 1.431e-05 & 49.72     & 49.72     & CSSFinder.py:1(\&lt;module\&gt;)  \\
  1       & 7.526e-05 & 7.526e-05 & 49.68     & 49.68     & CSSFinder.py:1059(main)           \\
  1       & 0.3098    & 0.3098    & 49.63     & 49.63     & CSSFinder.py:632(Gilbert)         \\
  1028    & 0.5381    & 0.0005234 & 38.2      & 0.03716   & CSSFinder.py:458(OptimizedFS)     \\\midrule
  411200  & 0.8332    & 2.026e-06 & 19.03     & 4.627e-05 & CSSFinder.py:289(RandomUdFS)      \\
  595516  & 0.67      & 1.125e-06 & 13.88     & 2.331e-05 & CSSFinder.py:228(RandomdFS)       \\
  411200  & 0.384     & 9.338e-07 & 13.17     & 3.203e-05 & CSSFinder.py:60(ExpanddFS)        \\
  822400  & 0.7256    & 8.823e-07 & 11.94     & 1.452e-05 & CSSFinder.py:45(Kronecker)        \\\midrule
  849257  & 10.3      & 1.213e-05 & 10.3      & 1.213e-05 & CSSFinder.py:63(Rotate)           \\
  1068026 & 6.535     & 6.118e-06 & 9.85      & 9.223e-06 & CSSFinder.py:29(Product)          \\
  1332780 & 2.17      & 1.628e-06 & 4.502     & 3.378e-06 & CSSFinder.py:21(Normalize)        \\
  1332780 & 2.247     & 1.686e-06 & 3.802     & 2.853e-06 & CSSFinder.py:33(Generate)         \\\midrule
  737264  & 0.4225    & 5.73e-07  & 2.548     & 3.456e-06 & CSSFinder.py:18(Outer)            \\
  595516  & 0.4642    & 7.794e-07 & 2.361     & 3.964e-06 & CSSFinder.py:26(Project)          \\
  1233601 & 0.8998    & 7.294e-07 & 1.165     & 9.447e-07 & CSSFinder.py:39(IdMatrix)         \\
  1       & 3.046e-06 & 3.046e-06 & 0.05277   & 0.05277   & CSSFinder.py:96(readmtx)          \\\midrule
  1       & 1.752e-06 & 1.752e-06 & 0.05277   & 0.05277   & CSSFinder.py:552(Initrho0)        \\
  1       & 4.597e-06 & 4.597e-06 & 0.002477  & 0.002477  & CSSFinder.py:1049(DisplayLogo)    \\
  1       & 5.189e-06 & 5.189e-06 & 0.0004394 & 0.0004394 & CSSFinder.py:954(DetectDim0)      \\
  1       & 1.628e-05 & 1.628e-05 & 2.526e-05 & 2.526e-05 & CSSFinder.py:556(Initrho1)        \\\midrule
  1       & 1.903e-06 & 1.903e-06 & 5.671e-06 & 5.671e-06 & CSSFinder.py:599(DefineSym)       \\
  40      & 3.038e-06 & 7.595e-08 & 3.038e-06 & 7.595e-08 & CSSFinder.py:192(writemtx)        \\
  1       & 1.102e-06 & 1.102e-06 & 2.846e-06 & 2.846e-06 & CSSFinder.py:624(DefineProj)      \\
  2       & 2.3e-07   & 1.15e-07  & 2.3e-07   & 1.15e-07  & CSSFinder.py:845(makeshortreport) \\\midrule
\end{tabular}
      \caption{Dane dotyczące pracy oryginalnej implementacji programu CSSFinder uzyskane przy pomocy programy cProfile. Tabela posiada oryginalne nazwy kolumn, nadane przez program snakeviz. Znaczenia kolumn, kolejno od lewej: \code{ncalls} - ilość wywołań funkcji. \code{tottime} - całkowity czas spędzony w ciele funkcji bez czasu spędzonego w wywołaniach do podfunkcji. \code{percall} - \code{totime} dzielone przez \code{ncalls}. \code{cumtime} - całkowity czas spędzony w wewnątrz funkcji i w wywołaniach podfunkcji. \code{percall} - \code{cumtime} dzielone przez \code{ncalls}. \code{filename:lineno(function)} - Plik, linia i nazwa funkcji.}
    \end{table}
    \FloatBarrier

    Z uzyskanych danych wynika że znakomitą większość (77\%\footnote{Wartość 77\% jak i
    wartości procentowe dalszej części tego akapitu zostały zaokrąglone do jedności, ze względu
    na małe znaczenie rzeczowe części ułamkowych.}) czasu pracy programu zajmuje funkcja
    \code{OptimizedFS()}. W jej wnętrzu 38\% czasu pochłania proces generowania losowych
    macierzy unitarnych, który w dużej mierze wykorzystuje mnożenia tensorowe (26\%).
    Poza funkcją \code{OptimizedFS()}, znaczący wpływ na czas wykonywania ma też funkcja
    `rotate()`, która pochłania około 21\% czasu działania programu. Kolejne 20\% czasu
    zajmuje funkcja \code{product()}, obliczająca odległość Hilberta-Schmidta pomiędzy
    dwoma stanami. Pozostałe wywołania mają stosunkowo marginalny wpływ na czas pracy i ich
    analiza na tym etapie nie niesie za sobą znaczących korzyści.

    Takie wyniki wskazują jednoznacznie że kluczowa dla czasu pracy programu jest tu maksymalizacja
    wydajności operacji macierzowych. Ten pozornie oczywisty wniosek wyznacza prosty
    kurs dalszych prac nad programem. W uzyskanych danych nie widać problemów z operacjami
    I/O\footnote{I/O - operacje wejścia wyjścia, w tym wypadku odczyt z i pisanie do plików.},
    a dla wielu aplikacji potrafią one stanowić poważne ograniczenie wydajności. W tym wypadku
    nie jest konieczne sięganie po rozwiązania takie jak asyncio czy wielowątkowość,
    które stosuje się w razie problemów z operacjami I/O.

    \subsection{Wstępne pomiary wydajności}


    Aby uzyskać dobrą bazę porównawczą, wykonałem serię pomiarów czasu pracy programu na
    macierzach $\rho_{0}1$, $\rho_{2}$ - $\rho_{6}$, przedstawionych na rysunkach \ref{rho-0}
    i \ref{rho-2-6}.

    Dane przekazywałem kolejno do programu z poleceniem działania w trybie 1 (full
    separability of an n-quDit state) do osiągnięcia 1000 korekcji lub do 2.000.000
    iteracji algorytmu, w zależności od tego co nastąpi szybciej. Dla wszystkich
    macierzy algorytm uzyskał 1000 korekcji i w żadnym przypadku nie osiągnął
    maksymalnej liczby iteracji. Dla każdej macierzy pomiar był powtarzany pięciokrotnie,
    a wyniki z pomiarów zostały uśrednione. Podczas obliczeń ziarno globalnego generatora
    liczb losowych biblioteki NumPy było ustawione na 0. Pomiary czasu pracy dotyczyły wyłącznie
    samego algorytmu\footnote{tj. funkcji `Gilbert()', nie biorą więc pod uwagę czasu
    pochłoniętego przez importowanie modułów, ładowanie danych itp. natomiast operacje
    pisania do plików które były wykonywane w obrębie tej funkcji są wliczane w czas
    pracy.}.

    \FloatBarrier
    \begin{figure}[ht]
      \centering
      \includegraphics[width=0.9\textwidth]{"resources/original_performance_tests.png"}
      \caption{Wyniki wstępnych testów wydajności oryginalnego kodu dla macierzy $\rho_{2}$ - $\rho
      _{6}$.}
      \label{pre-perf}
    \end{figure}
    \FloatBarrier

    Podczas testów zaobserwowałem interesujące zjawisko dotyczące wydajności dla macierzy
    $64\times64$. W przypadku takich rozmiarów danych biblioteka NumPy automatycznie
    decyduje o wykorzystaniu wielowątkowej implementacji mnożenia macierzowego. Niestety,
    daje to efekt odwrotny do zamierzonego - obliczenia zamiast przyspieszać zwalniają. Na
    rysunku \ref{pre-perf} zostały przedstawione czasy obliczeń dla macierzy $\rho_{2}$ -
    $\rho_{6}$ z domyślnym zachowaniem biblioteki.

    \FloatBarrier
    \begin{figure}[ht]
      \centering
      \includegraphics
      [width=0.9\textwidth]{"resources/original_performance_tests_locked.png"}
      \caption{Wyniki wstępnych testów wydajności oryginalnego kodu z zablokowaną liczbą wątków obliczeniowych dla macierzy $\rho
      _{2}$ - $\rho_{6}$.}
      \label{pre-perf-locked}
    \end{figure}
    \FloatBarrier

    Jeśli przy pomocy zmiennych środowiskowych ustawimy ilość wątków wykorzystywanych do
    obliczeń na 1 uzyskujemy znaczące skrócenie czasu obliczeń dla macierzy $64\times64$.
    Wyniki testów w takich warunkach zostały przedstawione na rysunku \ref{pre-perf-locked}.
    Dla macierzy w mniejszych rozmiarach nie odnotowałem różnicy w wydajności pomiędzy
    konfiguracją domyślną, a manualnie dostosowywaną. Warto dodać że ilość iteracji
    wykonywanych przez program nie zmienia się, różnica wynika wyłącznie z czasu trwania
    operacji arytmetycznych. Taki stan rzeczy najprawdopodobniej jest wynikiem
    dodatkowego obciążenia ze strony komunikacji i/lub synchronizacji między wątkami.

    \subsection{Re-implementacje}


    \subsubsection{ Python i NumPy }


    Pierwsza wykonana przeze mnie re-implementacja algorytmu, została napisana w języku Python,
    a do realizowania obliczeń na macierzach liczb zespolonych wykorzystywała bibliotekę
    NumPy. Podczas przepisywania podjąłem jednak dodatkowe wysiłki aby zastępować kod Pythona
    wywołaniami do funkcji zawartych w bibliotece NumPy. Ponieważ kluczowe dla
    wydajności fragmenty kodu tego pakietu są zaimplementowane w języku niższego poziomu,
    a następnie skompilowane kompilatorem optymalizującym, oferują znacznie wyższą
    wydajność niż analogiczny kod napisany w języku Python. Proces ten pozwolił mi również
    zapoznać się lepiej z charakterystyką programu i udoskonalić interfejs służący do komunikacji
    pomiędzy częścią główną, a samą implementacją (backend'em). Sam algorytm pozostał taki
    sam, natomiast konstrukcja kodu zmieniła się diametralnie, więc dogłębne analizy różnic
    byłyby nieczytelne, dlatego nie zostaną tutaj zawarte. W dalszej części pojawią się wyniki
    pomiarów wydajności.

    Pomiary czasu pracy były wykonywane przy użyciu macierzy $\rho_{2}$ - $\rho_{6}$.
    Dane przekazywałem kolejno do programu z poleceniem działania w trybie FSnQd\footnote{Tryb
    FSnQd jest odpowiednikiem trybu 1 (full separability of an n-quDit state) z oryginalnego
    kodu.} do osiągnięcia co najmniej 1000 korekcji lub do 2.000.000 iteracji algorytmu,
    w zależności od tego co nastąpi szybciej. Dla wszystkich macierzy algorytm uzyskał
    co najmniej 1000 korekcji i w żadnym przypadku nie osiągnął maksymalnej liczby
    iteracji. Dla każdej macierzy pomiar był powtarzany pięciokrotnie a wyniki zostały uśrednione.
    Podczas obliczeń ziarno domyślnego globalnego generatora liczb losowych biblioteki
    NumPy było ustawione na 0. Program działał z zablokowaną liczbą wątków
    obliczeniowych. Pomiary czasu pracy dotyczyły przede wszystkim samego algorytmu\footnote{Pomiary
    nie biorą więc pod uwagę czasu pochłoniętego przez importowanie modułów itp.,
    natomiast operacje wczytywania danych i pisania do plików są wliczane w czas pracy, ponieważ
    wbudowany w program mechanizm pomiaru czasu pracy rozpoczyna pomiar zanim dane zostaną
    załadowane.}.

    \FloatBarrier
    \begin{figure}[ht]
      \centering
      \includegraphics
      [width=0.9\textwidth]{"resources/python_and_numpy_performance_tests.png"}
      \caption{Wyniki testów wydajności alternatywnej implementacji Python z użyciem biblioteki NumPy dla macierzy $\rho
      _{2}$ - $\rho_{6}$.}
      \label{first-perf}
    \end{figure}
    \FloatBarrier

    Uzyskane wyniki zostały przedstawione na rysunku \ref{first-perf}. Wykres został utworzony
    przy pomocy biblioteki matplotlib\cite{Hunter:2007}.

    \subsubsection{ Python i NumPy z AOT }


    Następnym wykonanym przeze mnie krokiem było skompilowanie mojej implementacji
    korzystającej z NumPy do kodu maszynowego przy pomocy biblioteki Cython. Kod przeznaczony
    do takiej kompilacji nie musi być adnotowany dedykowanymi informacjami o typach.
    Zostanie on w tedy przetłumaczony na odpowiednie operacje w języku C/C++, a potem skompilowany
    do kodu maszynowego. Brak adnotacji powoduje niestety, że program zachowuje swoją dynamiczną
    naturę, charakterystyczną dla języka Python. Kompilacja pozwala jednak usunąć dodatkowy
    narzut na procesor ze strony interpretera. W takim scenariuszu spodziewać należy się,
    że zyski z kompilacji będą niewielkie, ale mogą wystąpić.

    Pomiary czasu pracy były wykonywane w taki sam sposób jak dla implementacji bez AOT.

    \FloatBarrier
    \begin{figure}[ht]
      \centering
      \includegraphics
      [width=0.9\textwidth]{"resources/python_and_numpy_and_aot_performance_tests.png"}
      \caption{Wyniki testów wydajności implementacji Python z użyciem biblioteki NumPy oraz pakietu Cython do kompilacji AOT dla macierzy $\rho
      _{2}$ - $\rho_{6}$.}
      \label{second-perf}
    \end{figure}
    \FloatBarrier

    Na rysunku \ref{second-perf} przedstawione zostały wyniki pomiarów czasu pracy skompilowanej
    wersji w języku Python bazującej na bibliotece NumPy wykorzystujące macierze $\rho_{2}$
    - $\rho_{6}$. Dodatkowa kompilacja nie poskutkowała widocznym skróceniem czasu pracy
    programu, jedynie wynik dla macierzy $64\times64$ różni się nieznacznie. Może to być
    spowodowane usunięciem szczątkowego obciążenia ze strony interpretera, które nie jest
    mierzalne podczas krótszych testów z mniejszymi macierzami. Natomiast możliwe jest
    również że ta różnica wynika z korzystniejszych warunków losowo zapewnionych przez
    system operacyjny.

    \begin{figure}[ht]
      \centering
      \includegraphics[width=0.9\textwidth]{"resources/benchmark_3/plot2.png"}
      \caption{Wyniki testów wydajności implementacji Python i NumPy Z AOT w porównaniu do implementacji bez AOT dla macierzy $\rho
      _{0}$.}
      \label{second-alt-perf}
    \end{figure}
    \FloatBarrier

    Wyniki dla testów wykonanych z użyciem macierzy $\rho_{0}$ widoczne na rysunku \ref{second-alt-perf}
    prowadzą do analogicznych wniosków co w przypadku macierzy $\rho_{2}$ - $\rho_{6}$ -
    pre-kompilacja całości z wykorzystaniem biblioteki Cython nie poprawia znacząco wydajności
    rozważanego kodu.

    \subsubsection{ Python i NumPy z JIT }
    \label{python_numpy_jit}

    Ostatnia stworzona przeze mnie re-implementacja w języku Python bazująca na
    bibliotece NumPy dodatkowo korzysta z kompilacji JIT. Pakiet Numba, który został
    wykorzystany do zrealizowania kompilacji JIT, posiada dwa tryby pracy. Pierwszy
    wykonuje kompilację na podstawie specjalnie dostarczonych przez programistę
    deklaracji typów dla funkcji podlegających kompilacji i jest wykonywany zaraz po
    rozpoczęciu pracy programu\footnote{ang. eager (compilation) - niecierpliwa (kompilacja).}.
    Drugi polega na śledzeniu typów wejściowych i wyjściowych funkcji i automatycznie kompiluje
    funkcję dla tych typów danych które są odpowiednio często używane używane\footnote{ang.
    lazy (compilation) - leniwa (kompilacja).}.

    Ponadto, Numba posiada dodatkowe parametry kompilacji, które można przekazać do funkcji
    \code{numba.jit}. Jednym z nich, posiadającym szczególnie duży wpływ na wydajność, flaga
    \code{nopython}. Tryb \code{nopython=True} oferuje znacznie większe możliwości
    optymalizacji i potencjalnie lepszą wydajność. Niestety nie wszystkie funkcje dostępne
    w bibliotece NumPy są akceptowane przez kompilator JIT pakietu Numba w trybie \code{nopython=True}.
    Do niekompatybilnych należy między innymi funkcja tensordot która implementuje mnożenie
    tensorowe. Wspomniana funkcja może zostać skompilowana tylko w trybie obiektowym (\code{nopython=False}),
    który po kompilacji zachowuje dynamiczną naturę Pythona. Niestety, brak możliwości
    skompilowania funkcji używającej tensordot powoduje również brak możliwości
    skompilowania funkcji wyżej w drzewie wywołań. W efekcie znacząca część implementacji
    używającej JIT musi używać trybu obiektowego.

    Pomiary czasu pracy były wykonywane w taki sam sposób jak dla implementacji bez JIT.

    \FloatBarrier
    \begin{figure}[ht]
      \centering
      \includegraphics
      [width=0.9\textwidth]{"resources/python_and_numpy_and_jit_performance_tests.png"}
      \caption{Wyniki testów wydajności implementacji w języku Python z użyciem biblioteki NumPy i pakietu Numba do kompilacji JIT dla macierzy $\rho
      _{2}$ - $\rho_{6}$.}
      \label{third-perf}
    \end{figure}
    \FloatBarrier

    Na rysunku \ref{third-perf} przedstawione zostały wyniki uzyskane podczas pomiarów
    czasu pracy implementacji z JIT, w zależności od rozmiaru macierzy gęstości. Kod
    który wykorzystywał kompilację JIT oferował podczas testów dwu-czterokrotnie (w zależności
    od rozmiaru macierzy testowej) większą wydajność niż kod bez niej. Tak znaczącą poprawę
    implementacja zawdzięcza prawdopodobnie temu, że kompilator JIT może specjalizować
    kod dla dokładnie jednej platformy, korzystając z całego spektrum jej możliwości.
    Dotyczy to na przykład instrukcji SIMD, takich jak AVX512, które są dostępne w
    procesorze użytym do testów, ale wiele wciąż popularnych procesorów ich nie posiada.
    Wymusza to, przy kompilacji AOT, zastąpienie tych instrukcji innymi szerzej
    dostępnymi, aby zmaksymalizować przenośność kodu. Dodatkowo kompilator może brać pod
    uwagę inne charakterystyczne cechy konkretnych architektur. Te dodatkowe informacje i
    możliwość dodatkowej specjalizacji kodu czynią kompilację JIT bardzo potężnym narzędziem

    \subsubsection{ Rust i Ndaray}


    Aby uczynić to porównanie jak najpełniejszym, podjąłem również wysiłek zaimplementowania
    części obliczeniowej programu w języku Rust. Język ten wybrałem z kilku względów. Posiada
    on pełną infrastrukturę pozwalającą w łatwy sposób kompilować programy i biblioteki wykorzystujące
    stworzone przez innych programistów rozwiązania. Daje mu to znaczącą przewagę nad
    językami takimi jak C/C++ które wymagają aby bardziej skomplikowane projekty
    samodzielnie skompletowały systemy budowania opierającego się na rozwiązaniach
    podmiotów trzecich, takich jak CMake, szczególnie jeśli chcą być dostępne na wielu
    platformach. Ponadto konkurencja nie posiada ujednoliconego standardu pozwalającego na
    łatwe uzyskanie odstępu do bibliotek otwartoźródłowych, Rust natomiast taki system
    posiada. W efekcie w łatwy sposób mogłem dołączyć gotowe rozwiązania pozwalające na prowadzenie
    obliczeń na macierzach liczb zespolonych. W efekcie cały proces wstępnej
    konfiguracji sprowadził się do około godziny, co stanowi wyśmienity wynik, biorąc
    pod uwagę, że przed podejściem do tego projektu nie miałem żadnej praktycznej styczności
    z tym językiem programowania. Dodatkową zaletą tego języka jest automatyczny system
    zarządzania pamięcią oparty na koncepcji własności (ang. ownership), który usuwa
    konieczność manualnego zarządzania pamięcią, jednocześnie bez konieczności wprowadzania
    mechanizmu liczenia referencji i dedykowanego automatycznego `odśmiecacza' (ang. garbage
    collector) które to są częstym źródłem problemów z wydajnością i użyciem pamięci.

    Pomiary czasu pracy implementacji w języku Rust były wykonywane przy użyciu macierzy
    $\rho_{2}$ - $\rho_{6}$. Dane przekazywałem kolejno do programu z poleceniem działania
    w trybie FSnQd do osiągnięcia co najmniej 1000 korekcji lub do 2.000.000 iteracji
    algorytmu, w zależności od tego co nastąpi szybciej. Dla wszystkich macierzy
    algorytm uzyskał co najmniej 1000 korekcji i w żadnym przypadku nie osiągnął
    maksymalnej liczby iteracji. Dla każdej macierzy pomiar był powtarzany pięciokrotnie
    a wyniki zostały uśrednione. Pomiary czasu pracy dotyczyły przede wszystkim samego
    algorytmu\footnote{Nie biorą więc pod uwagę czasu pochłoniętego przez importowanie modułów
    itp., natomiast operacje wczytywania danych i pisania do plików są wliczane w czas pracy.}.

    \FloatBarrier
    \begin{figure}[ht]
      \centering
      \includegraphics[width=0.9\textwidth]{"resources/rust_performance_tests.png"}
      \caption{Wyniki testów wydajności implementacji w języku Rust dla macierzy $\rho_{2}$ - $\rho
      _{6}$.}
      \label{fourth-perf}
    \end{figure}
    \FloatBarrier

    Na rysunku \ref{fourth-perf} zaprezentowane zostały wyniki pomiarów czasu pracy
    implementacja w języku Rust. Prezentuje ona znacząco lepsze wyniki dla małych macierzy
    oraz znacznie gorsze wyniki dla dużych macierzy. Jest to prawdopodobnie spowodowane
    tym, że sama implementacja mnożenia macierzowego nie jest wyspecjalizowana, aby
    wykorzystywać maksimum możliwości procesora na którym jest wykonywana, w przeciwieństwie
    do na przykład biblioteki NumPy, które wewnętrznie wykorzystuje bibliotekę OpenBLAS\cite{NumPy_Doc}.
    W efekcie nie czerpie ona korzyści z instrukcji SIMD, takich jak AVX512.

    \subsubsection{ Rust i Ndaray z OpenBLAS }


    Biblioteka Ndarray, która jest sercem implementacji w języku Rust, posiada przełącznik
    funkcjonalności\footnote{ang. feature switch} który pozwala wykorzystać funkcje
    zawarte w bibliotece OpenBLAS jako implementację mnożenia macierzowego. O ile kompilacja
    dla wszystkich platform które ma wspierać CSSFinder (Windows, Linux i MacOS) jest
    poza moim zasięgiem, to uznałem, że warto zweryfikować jakie efekty daje
    wykorzystanie tej funkcjonalności w środowisku laboratoryjnym.

    Pomiary czasu pracy były wykonywane w taki sam sposób jak dla implementacji która
    nie korzystała z OpenBLAS.

    \FloatBarrier
    \begin{figure}[ht]
      \centering
      \includegraphics[width=0.9\textwidth]{"resources/rust_blas_perf_tests.png"}
      \caption{Wyniki testów wydajności implementacji w języku Rust z użyciem biblioteki OpenBLAS dla macierzy $\rho
      _{2}$ - $\rho_{6}$.}
      \label{fifth-perf}
    \end{figure}
    \FloatBarrier

    Wykorzystanie biblioteki OpenBLAS poskutkowało znaczącym wzrostem wydajności,
    przekraczającym możliwości oryginalnej implementacji. Wyniki te zostały przedstawione
    na rysunku \ref{fifth-perf}. Kod tutaj omawiany ustępuje jedynie implementacji z
    sekcji \ref{python_numpy_jit}, wykorzystującej JIT. Jednocześnie sprawuje się on
    gorzej dla małych macierzy niż wariant bez OpenBLAS.

    \subsection{Precyzja obliczeń}


    Oryginalny program, jak i re-implementacje które pojawiły się powyżej, posługiwały
    się liczbami zespolonymi na bazie liczb zmiennoprzecinkowych podwójnej precyzji. Jedna
    taka liczba zajmuje 64 bity. Jednak w wielu przypadkach taka precyzja obliczeń nie
    jest konieczna do uzyskania poprawnych wyników. Podstawową zaletą wykorzystania liczb
    zmiennoprzecinkowych pojedynczej precyzji, czyli 32 bitowych, jest zmniejszenie rozmiaru
    macierzy. Pozwala na umieszczenie większej części macierzy w pamięci podręcznej
    procesora. Dodatkowo zwiększa to przepustowość obliczeń wykorzystujących instrukcje SIMD,
    ponieważ wykorzystują one rejestry o stałych rozmiarach (128, 256, 512 bitów) które
    mogą na ogół pomieścić dwukrotnie więcej liczb 32 bitowych niż 64 bitowych. Pozwala to
    oczekiwać że obliczenia wykorzystujące liczby zmiennoprzecinkowe pojedynczej precyzji
    będą trwały krócej.

    Tworzony przeze mnie kod od początku powstawał z zamysłem umożliwienia wykorzystania
    liczb zmiennoprzecinkowych o różnych precyzjach, dlatego transformacja ta była dość
    prosta. W języku Python, wykorzystując bibliotekę NumPy przejście na liczby
    pojedynczej precyzji wymagało prawie każdorazowego deklarowania że wynik operacji ma
    posiadać typ complex64 (cały czas mówimy o liczbach zespolonych które składają się z
    dwóch wartości zmiennoprzecinkowych). Nie wszystkie operacje które przyjmują parametr
    określający typ wejściowy są akceptowane przez kompilator JIT biblioteki Numba gdy jest
    on przekazywany. To ograniczenie można obejść wykonując zmianę typu jako osobną
    operację przy pomocy metody \code{astype()}.

    Warto tutaj zaznaczyć że wszystkie implementacje w języku Python powstają ze wspólnego
    szablony który był ewaluowany przez bibliotekę Jinja2 do różnych wariantów kodu, w
    zależności od tego jakie parametry były do niego przekazywane. Pozwoliło to uniknąć wielokrotnego
    pisania wspólnych fragmentów kodu, a elementy unikalne są dodawane warunkowo. Zastosowanie
    introspekcji do konstruowania odpowiedniego kodu w trakcie wykonywania programu mogłoby
    w znaczący sposób obniżyć wydajność, dlatego zdecydowałem się sięgnąć po system
    bardziej statyczny, który na pewno nie wpływał na czas pracy programu.

    W przypadku języka Rust, posiada on dedykowany konstrukt składniowy pozwalający na
    deklarowanie funkcji w oparciu o symbole zastępcze wobec których stawia się zbiór
    wymagań dotyczących wspieranych interfejsów. W efekcie funkcja może zostać wyspecjalizowana
    żeby akceptować zarówno liczby zespolone skonstruowane z liczb zmiennoprzecinkowych pojedynczej
    jak i podwójnej precyzji. Pozwoliło to uniknąć sięgania po zewnętrzne mechanizmy do
    tworzenia szablonów, tak jak było to konieczne w języku Python.

    \subsection{Pomiary z pojedynczą precyzją}


    \subsubsection{ Python i NumPy }


    \FloatBarrier
    \begin{figure}[ht]
      \centering
      \includegraphics
      [width=0.9\textwidth]{"resources/python_and_numpy_single_tests.png"}
      \caption{Wyniki testów wydajności implementacji w języku Python z użyciem biblioteki NumPy dla macierzy $\rho
      _{2}$ - $\rho_{6}$ i obliczeniami pojedynczej precyzji.}
      \label{sp-numpy-perf}
    \end{figure}
    \FloatBarrier

    Na wykresie \ref{sp-numpy-perf} przedstawiłem wyniki wydajności dla implementacji
    napisanej w języku Python wykorzystującej bibliotekę NumPy do przeprowadzania
    obliczeń na macierzach liczb zespolonych pojedynczej precyzji. W przypadku mniejszych
    macierzy ($4\times4$, $8\times8$, $16\times16$) żadne różnice w czasie pracy, względem
    wariantu opartego na liczbach podwójnej precyzji, są minimalne,. Dzieje się tak prawdopodobnie
    dlatego, że macierze te są na tyle niewielkie (do 4KB) że mieszczą się w pamięci cache
    L1 procesora\footnote{Wykorzystywany tutaj Ryzen 9 7950X posiada $32\times16$KB cache
    L1}, więc wymnażanie ich jest procesem bardzo szybkim. W momencie kiedy docieramy do
    macierzy $32\times32$ co również można wytłumaczyć odwołując się do pojemności
    pamięci cache procesora. Macierze podwójnej precyzji zajmują dokładnie 16KB ($32\times
    32\times2\times 8$), natomiast dostęp do tej pamięci nie jest ekskluzywny dla
    jednego procesu, nie może on więc korzystać z całych 16KB. W efekcie część danych
    przebywa poza pamięcią cache. Natomiast macierze wykorzystujące liczby pojedynczej precyzji
    zajmują tylko 8KB. Można się więc spodziewać że większość czasu spędzają one w
    pamięciach L1 i L2, co pozwala przyspieszyć obliczenia. Dodatkowo mniejszy rozmiar
    liczb pojedynczej precyzji pozwala dwukrotnie zwiększyć przepustowość instrukcji
    SIMD, co prawdopodobnie odgrywa również bardzo istotną rolę, szczególnie w przypadku
    macierzy $64\times64$, dla których obliczenia przyspieszają znacznie bardziej niż w przypadku
    mniejszych macierzy.

    \subsubsection{ Python i NumPy z AOT }


    \FloatBarrier
    \begin{figure}[ht]
      \centering
      \includegraphics
      [width=0.9\textwidth]{"resources/python_and_numpy_and_aot_single_tests.png"}
      \caption{Wyniki testów wydajności implementacji w języku Python z użyciem biblioteki NumPy i pakietu Cython do kompilacji AOT dla macierzy $\rho
      _{2}$ - $\rho_{6}$ i obliczeniami pojedynczej precyzji.}
      \label{sp-numpy-cython-perf}
    \end{figure}
    \FloatBarrier

    Wyniki dla wariantu pre-kompilowanego przy pomocy biblioteki Cython nie różnią się
    znacząco od wariantu nie pre-kompilowanego, podobnie jak w przypadku obliczeń podwójnej
    precyzji, zostały one zaprezentowane na rysunku \ref{sp-numpy-cython-perf}.

    \newpage


    \subsubsection{ Python i NumPy z JIT }


    \FloatBarrier
    \begin{figure}[ht]
      \centering
      \includegraphics
      [width=0.9\textwidth]{"resources/python_and_numpy_and_jit_single_tests.png"}
      \caption{Wyniki testów wydajności implementacji w języku Python z użyciem biblioteki NumPy i pakietu Numba do kompilacji JIT dla macierzy $\rho
      _{2}$ - $\rho_{6}$ i obliczeniami pojedynczej precyzji.}
      \label{sp-numpy-jit-perf}
    \end{figure}
    \FloatBarrier

    W przypadku wariantu wykorzystującego kompilację JIT, zysk czasowy jest minimalny
    lub wręcz nie występuje. Ciężko mi wytłumaczyć dlaczego tak się dzieje, możliwe, że coś
    powoduje że obliczenia korzystają z tej samej implementacji.

    \subsubsection{ Rust i Ndaray}


    \FloatBarrier
    \begin{figure}[ht]
      \centering
      \includegraphics[width=0.9\textwidth]{"resources/rust_single_tests.png"}
      \caption{Wyniki testów wydajności implementacji w języku Rust z użyciem biblioteki Ndarray dla macierzy $\rho
      _{2}$ - $\rho_{6}$ i obliczeniami pojedynczej precyzji.}
      \label{sp-rust-perf}
    \end{figure}
    \FloatBarrier

    Implementacja w języku Rust wykorzystująca bibliotekę Ndarray prezentuje najlepszą
    wydajność podczas obliczeń na małych macierzach, do $16\times16$ włącznie. Wynika to
    najprawdopodobniej z braku dodatkowego obciążenia ze strony interpretera, którego wewnętrzne
    operacje wprowadzają dodatkowe informacje do pamięci cache, tym samym wypierając z
    niej macierze na których są prowadzone obliczenia. W przypadku języka interpretowanego,
    CPU musi wykonywać o znacznie więcej instrukcji, ponieważ każda operacja w języku wysokiego
    poziomu musi zostać załadowana, po czym odpowiednia akcja musi zostać wybrana i
    dopiero wykonana. Tak długo jak kluczowe dla wydajności nie jest to ile zajmują mnożenia
    macierzowe, tak długo implementacja w języku Rust będzie szybsza.

    \subsubsection{ Rust i Ndaray z OpenBLAS }


    \section{Wyniki}


    \section{Dyskusja}
  \end{sloppypar}
  \newpage
  \begin{sloppypar}
    \medskip


    \printbibliography
    [heading=bibintoc, title={Odwołania}]
  \end{sloppypar}
\end{document}